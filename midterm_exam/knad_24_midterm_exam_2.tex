% arara: xelatex
\documentclass[12pt]{article}

% \usepackage{physics}


\usepackage{hyperref}
\hypersetup{
    colorlinks=true,
    linkcolor=blue,
    filecolor=magenta,      
    urlcolor=cyan,
    pdftitle={Overleaf Example},
    pdfpagemode=FullScreen,
    }

\usepackage{tikzducks}

\usepackage{tikz} % картинки в tikz
\usepackage{microtype} % свешивание пунктуации

\usepackage{array} % для столбцов фиксированной ширины

\usepackage{indentfirst} % отступ в первом параграфе

\usepackage{sectsty} % для центрирования названий частей
\allsectionsfont{\centering}

\usepackage{amsmath, amsfonts, amssymb} % куча стандартных математических плюшек

\usepackage{comment}

\usepackage[top=2cm, left=1.2cm, right=1.2cm, bottom=2cm]{geometry} % размер текста на странице

\usepackage{lastpage} % чтобы узнать номер последней страницы

\usepackage{enumitem} % дополнительные плюшки для списков
%  например \begin{enumerate}[resume] позволяет продолжить нумерацию в новом списке
\usepackage{caption}

\usepackage{url} % to use \url{link to web}


\newcommand{\smallduck}{\begin{tikzpicture}[scale=0.3]
    \duck[
        cape=black,
        hat=black,
        mask=black
    ]
    \end{tikzpicture}}

\usepackage{fancyhdr} % весёлые колонтитулы
\pagestyle{fancy}
\lhead{Теория вероятностей и статистика: кнад}
\chead{}
\rhead{Контрольная 1, версия 2}
\lfoot{}
\cfoot{Да пребудет с тобой сила!}
\rfoot{}

\renewcommand{\headrulewidth}{0.4pt}
\renewcommand{\footrulewidth}{0.4pt}

\usepackage{tcolorbox} % рамочки!

\usepackage{todonotes} % для вставки в документ заметок о том, что осталось сделать
% \todo{Здесь надо коэффициенты исправить}
% \missingfigure{Здесь будет Последний день Помпеи}
% \listoftodos - печатает все поставленные \todo'шки


% более красивые таблицы
\usepackage{booktabs}
% заповеди из докупентации:
% 1. Не используйте вертикальные линни
% 2. Не используйте двойные линии
% 3. Единицы измерения - в шапку таблицы
% 4. Не сокращайте .1 вместо 0.1
% 5. Повторяющееся значение повторяйте, а не говорите "то же"


\setcounter{MaxMatrixCols}{20}
% by crazy default pmatrix supports only 10 cols :)


\usepackage{fontspec}
\usepackage{libertine}
\usepackage{polyglossia}

\setmainlanguage{russian}
\setotherlanguages{english}

% download "Linux Libertine" fonts:
% http://www.linuxlibertine.org/index.php?id=91&L=1
% \setmainfont{Linux Libertine O} % or Helvetica, Arial, Cambria
% why do we need \newfontfamily:
% http://tex.stackexchange.com/questions/91507/
% \newfontfamily{\cyrillicfonttt}{Linux Libertine O}

\AddEnumerateCounter{\asbuk}{\russian@alph}{щ} % для списков с русскими буквами
\setlist[enumerate, 2]{label=\asbuk*),ref=\asbuk*}

%% эконометрические сокращения
\DeclareMathOperator{\Cov}{\mathbb{C}ov}
\DeclareMathOperator{\pCorr}{\mathrm{pCorr}}
\DeclareMathOperator{\Corr}{\mathbb{C}orr}
\DeclareMathOperator{\Var}{\mathbb{V}ar}
\DeclareMathOperator{\col}{col}
\DeclareMathOperator{\row}{row}

\let\P\relax
\DeclareMathOperator{\P}{\mathbb{P}}

\let\H\relax
\DeclareMathOperator{\H}{\mathbb{H}}

\DeclareMathOperator{\CE}{\mathrm{CE}}



\DeclareMathOperator{\E}{\mathbb{E}}
% \DeclareMathOperator{\tr}{trace}
\DeclareMathOperator{\card}{card}

\DeclareMathOperator{\Convex}{Convex}

\newcommand \cN{\mathcal{N}}
\newcommand \RR{\mathbb{R}}
\newcommand \NN{\mathbb{N}}


\usepackage{mathtools}
\DeclarePairedDelimiter{\norm}{\lVert}{\rVert}
\DeclarePairedDelimiter{\abs}{\lvert}{\rvert}
\DeclarePairedDelimiter{\scalp}{\langle}{\rangle}
\DeclarePairedDelimiter{\ceil}{\lceil}{\rceil}



\begin{document}

\begin{enumerate}
    \item {[10]} Величины $(X_n)$ независимы и одинаково распределены с плостностью $f(x) = 2 - 2x$ на отрезке $[0;1]$.
    \begin{enumerate}
        \item {[2]} Найди число $c$ такое, что величина $X_1$ не превышает $c$ с вероятностью $1/4$.
        \item {[2]} Найдите энтропию случайной величины $X_1$.
        \item {[2]} Найдите совместную энтропию пары $(X_1, X_2)$.
        \item {[4]} Найдите функцию плотности величины $Y = \exp(X)$.
    \end{enumerate}

    \item {[10]} Совместная функция плотности вектора $(X, Y)$ равна $f(x, y) = 6xy^2$ на квадрате $[0;1] \times [0;1]$ и $0$ за его пределами. 
    \begin{enumerate}
        \item {[2]} Найдите вероятность $\P(X > Y > X/2)$. 
        \item {[6]} Найдите $\E(X)$, $\E(XY)$, $\Cov(X, Y)$.
        \item {[2]} Зависимы ли величины $X$ и $Y$?
    \end{enumerate}
    
    В этой задаче в этот раз можно оставить ответ в виде аккуратно выписанного интеграла :)
   

    \item {[10]} Глеб Жеглов каждый день ловит одного преступника.
    С вероятностью $0.2$ на преступный путь вместо пойманного преступника встают два новых горожанина. 
    Изначально в городе $1$ преступник. 
    
    Обозначим за $T$ день поимки последнего преступника в городе. 

    \begin{enumerate}
      \item {[2]} Найдите $\P(T = 4)$.
      \item {[5]} Найдите ожидание $\E(T)$ и дисперсию $\Var(T)$.
      \item {[3]} Найдите функцию $m(u)$, производяющую моменты $T$.
    \end{enumerate}

    
    
    \item {[10]} Эксперт пытается определить, говорит ли подозреваемый правду, с помощью детектора лжи.
    Если подозреваемый говорит правду, то эксперт ошибочно выявляет ложь с вероятностью $0.2$. 
    Если подозреваемый обманывает, то эксперт выявляет ложь с вероятностью $0.9$. 
    В деле об одиночном нападении подозревают десять человек, один из которых виновен и будет лгать, 
    остальные — невиновны и говорят правду.
    
    Эксперт выбрал двух подозреваемых, Алису и Боба, наугад. 
    \begin{enumerate}
        \item {[3]} Какова вероятность того, что детектор покажет, что Алиса лжёт?
        \item {[3]} Какова вероятность того, что Алиса невиновна, если детектор показал, что она лжёт?
        \item {[4]} Какова вероятность того, что Алиса невиновна, если детектор показал, что она лжёт, и показал, что Боб говорит правду.
    \end{enumerate}


    \item {[10]} Алиса выбирает один раз равномерно распределённой число от $0$ до $3$. 
    Если число оказывается больше $1$, то она заменяет его на $2$, а затем выплачивает получившуюся величину Бобу.
    Обозначим $X$ — выигрыш Боба.
    \begin{enumerate}
        \item {[5]} Найдите $\E(X)$, $\Var(X)$.
        \item {[5]} Найдите функцию распределения $F(x)$ величины $X$ и постройте её график.
    \end{enumerate}


    \item {[10]} У Илона Маска $101$ золотая монета. 
    Он подкидывает каждую из них. 
    Пусть $X$ монеток выпали решкой вверх. 
    Затем Илон повторно подкидывает те монетки, которые лежат орлом вверх. 
    После второго раунда в общей сложности $Y$ монеток лежат решкой вверх.

    \begin{enumerate}
        \item {[4]} Найдите ожидание $\E(Y)$, дисперсию $\Var(Y)$.
        \item {[3]} Найдите наиболее вероятное значение величины $Y$.
        \item {[3]} Найдите $\Cov(X, Y)$.
    \end{enumerate}
\end{enumerate}

\end{document}

% здесь проектируемая часть


