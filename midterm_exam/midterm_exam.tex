% arara: xelatex
\documentclass[12pt]{article}

% \usepackage{physics}


\usepackage{hyperref}
\hypersetup{
    colorlinks=true,
    linkcolor=blue,
    filecolor=magenta,      
    urlcolor=cyan,
    pdftitle={Overleaf Example},
    pdfpagemode=FullScreen,
    }

\usepackage{tikzducks}

\usepackage{tikz} % картинки в tikz
\usepackage{microtype} % свешивание пунктуации

\usepackage{array} % для столбцов фиксированной ширины

\usepackage{indentfirst} % отступ в первом параграфе

\usepackage{sectsty} % для центрирования названий частей
\allsectionsfont{\centering}

\usepackage{amsmath, amsfonts, amssymb} % куча стандартных математических плюшек

\usepackage{comment}

\usepackage[top=2cm, left=1.2cm, right=1.2cm, bottom=2cm]{geometry} % размер текста на странице

\usepackage{lastpage} % чтобы узнать номер последней страницы

\usepackage{enumitem} % дополнительные плюшки для списков
%  например \begin{enumerate}[resume] позволяет продолжить нумерацию в новом списке
\usepackage{caption}

\usepackage{url} % to use \url{link to web}


\newcommand{\smallduck}{\begin{tikzpicture}[scale=0.3]
    \duck[
        cape=black,
        hat=black,
        mask=black
    ]
    \end{tikzpicture}}

\usepackage{fancyhdr} % весёлые колонтитулы
\pagestyle{fancy}
\lhead{Теория вероятностей и статистика: кнад}
\chead{}
\rhead{Контрольная 1, 2024-11-14}
\lfoot{}
\cfoot{Да пребудет с тобой сила!}
\rfoot{}

\renewcommand{\headrulewidth}{0.4pt}
\renewcommand{\footrulewidth}{0.4pt}

\usepackage{tcolorbox} % рамочки!

\usepackage{todonotes} % для вставки в документ заметок о том, что осталось сделать
% \todo{Здесь надо коэффициенты исправить}
% \missingfigure{Здесь будет Последний день Помпеи}
% \listoftodos - печатает все поставленные \todo'шки


% более красивые таблицы
\usepackage{booktabs}
% заповеди из докупентации:
% 1. Не используйте вертикальные линни
% 2. Не используйте двойные линии
% 3. Единицы измерения - в шапку таблицы
% 4. Не сокращайте .1 вместо 0.1
% 5. Повторяющееся значение повторяйте, а не говорите "то же"


\setcounter{MaxMatrixCols}{20}
% by crazy default pmatrix supports only 10 cols :)


\usepackage{fontspec}
\usepackage{libertine}
\usepackage{polyglossia}

\setmainlanguage{russian}
\setotherlanguages{english}

% download "Linux Libertine" fonts:
% http://www.linuxlibertine.org/index.php?id=91&L=1
% \setmainfont{Linux Libertine O} % or Helvetica, Arial, Cambria
% why do we need \newfontfamily:
% http://tex.stackexchange.com/questions/91507/
% \newfontfamily{\cyrillicfonttt}{Linux Libertine O}

\AddEnumerateCounter{\asbuk}{\russian@alph}{щ} % для списков с русскими буквами
\setlist[enumerate, 2]{label=\asbuk*),ref=\asbuk*}

%% эконометрические сокращения
\DeclareMathOperator{\Cov}{\mathbb{C}ov}
\DeclareMathOperator{\pCorr}{\mathrm{pCorr}}
\DeclareMathOperator{\Corr}{\mathbb{C}orr}
\DeclareMathOperator{\Var}{\mathbb{V}ar}
\DeclareMathOperator{\col}{col}
\DeclareMathOperator{\row}{row}

\let\P\relax
\DeclareMathOperator{\P}{\mathbb{P}}

\let\H\relax
\DeclareMathOperator{\H}{\mathbb{H}}

\DeclareMathOperator{\CE}{\mathrm{CE}}



\DeclareMathOperator{\E}{\mathbb{E}}
% \DeclareMathOperator{\tr}{trace}
\DeclareMathOperator{\card}{card}

\DeclareMathOperator{\Convex}{Convex}

\newcommand \cN{\mathcal{N}}
\newcommand \RR{\mathbb{R}}
\newcommand \NN{\mathbb{N}}


\usepackage{mathtools}
\DeclarePairedDelimiter{\norm}{\lVert}{\rVert}
\DeclarePairedDelimiter{\abs}{\lvert}{\rvert}
\DeclarePairedDelimiter{\scalp}{\langle}{\rangle}
\DeclarePairedDelimiter{\ceil}{\lceil}{\rceil}



\begin{document}

\begin{enumerate}
    \item {[10]} Величины $(X_n)$ независимы и одинаково распределены с плостностью $f(x) = \max\{0, 1 - \abs{x}\}$.
    \begin{enumerate}
        \item {[2]} Найди число $c$ такое, что величина $X_1$ превышает $c$ с вероятностью $1/4$.
        \item {[2]} Найдите функцию $m_X(t)$ производящую моменты величины $X_n$.
        \item {[2]} Найдите функцию $m_R(t)$ производящую моменты величины $R = X_1 + X_2 + \dots + X_{10} + 2024$.
        \item {[4]} Найдите функцию плотности величины $Y = \ln (X^2)$.
    \end{enumerate}

    \begin{enumerate}
        \item $c = 1 - \sqrt{2}/2$;
        \item 
        \item $(m_X(t))^{10}\cdot \exp(2024t)$;
        \item Заметим, что функция плотности $\abs{X}$ равна $2 - 2x$ на отрезке $[0;1]$ (все отрицательные значения отражаем вправо).
        И далее, $\abs{X} = \exp(Y/2)$, отсюда $f_Y(y) = (2 - 2exp(y/2)) \frac{1}{2}\exp(y/2) = \exp(y/2) - \exp(y)$ при $y<0$.
    \end{enumerate}

    \item {[10]} Совместная функция плотности вектора $(X, Y)$ равна $f(x, y) = 2x^3 + y$ на квадрате $[0;1] \times [0;1]$ и $0$ за его пределами. 
    \begin{enumerate}
        \item {[2]} Найдите вероятность $\P(X > Y,\, Y > 0.5)$. 
        \item {[6]} Найдите $\E(X)$, $\E(XY)$, $\Cov(X, Y)$.
        \item {[2]} Зависимы ли величины $X$ и $Y$?
    \end{enumerate}
    
    \begin{enumerate}
        \item $\E(X) = 13/20$, $\E(Y) = 7/12$, $\E(XY) = 11/30$, $\Cov(X, Y) = -1 / 80$,
        \item Величины зависимы так как функция $f(x, y)$ не раскладывается в произведение $f_X(x)$ на $f_Y(y)$.
    \end{enumerate}
   


    \item {[10]} Дональд Трамп подкидывает пару стандартных игральных кубиков до тех пор, пока одновременно не выпадет две шестёрки.
    Обозначим $N$ — общее количество бросков кубиков (бросок пары считаем за два броска кубика), а $S$ — общее количество выпавших шестёрок во всех бросках. 
    \begin{enumerate}
        \item {[2]} Найдите вероятности $\P(N = 6)$ и $\P(S = 3)$.
        \item {[4]} Найдите ожидание $\E(N)$ и дисперсию $\Var(N)$.
        \item {[4]} Найдите энтропии $\H(N)$ и $\H(S)$.
    \end{enumerate}

    \begin{enumerate}
        \item $\P(N = 6) = (35/36)^2 (1/36)$, $\P(S = 3) = 10/11 \cdot 1/11$ 
        \item $\E(N) = 2\cdot 1/p = 72$, $\Var(N) = 4 \cdot (1 - p)/p^2 = 5040$;
        \item 
    \end{enumerate}
    
    

    \item {[10]} Подсудимый виновен с некоторой вероятностью $p$.
    Независимо друг от друга и от виновности подсудимого, каждый из 12 присяжных проголосует за верное решение (виновен или не виновен) с вероятностью $3/5$
    и за ошибочное — с вероятностью $2/5$.
    \begin{enumerate}
        \item {[4]} Найдите ожидание и дисперсию количества присяжных, голосующих за виновность. 
        \item {[2]} Найдите вероятность того, что ровно $7$ присяжных проголосуют за виновность. 
        \item {[2]} Найдите наиболее вероятное число верно проголосовавших присяжных.
        \item {[2]} Найдите вероятность того, что подсудимый виновен, если ровно $7$ присяжных проголосовали за его виновность. 
    \end{enumerate}

    \begin{enumerate}
        \item     $\E(N) = p (12 \cdot 3/5) + (1 - p)(12 \cdot 2/5)$;
        $\E(N^2) = p(12 \cdot 3/5 \cdot 2/5 + (12 \cdot 3/5)^2) + (1-p) (12 \cdot 3/5 \cdot 2/5 + (12 \cdot 2/5)^2)$;

        \item $\P(N = 7) = p C_{12}^7 (3/5)^7 (2/5)^5 + (1 - p) C_{12}^7 (3/5)^5 (2/5)^7$;
        \item $\arg\max \P(G = g) = 7$, где $\P(G = g) = C_{12}^g (3/5)^g (2/5)^{12-g}$;
        \item $\P(A \mid N= 7) =  p C_{12}^7 (3/5)^7 (2/5)^5 / (p C_{12}^7 (3/5)^7 (2/5)^5 + (1 - p) C_{12}^7 (3/5)^5 (2/5)^7)$
    \end{enumerate}


    

    \item {[10]} Алиса подбрасывает правильную монетку. Если монетка выпадет орлом, то Алиса выплачивает Бобу один рубль. 
    Если монетка выпадает решкой, то Алиса выплачивает Бобу случайную сумму равноверно распределённую от $0$ до $2$ рублей. 
    Обозначим $X$ — выигрыш Боба.
    \begin{enumerate}
        \item {[5]} Найдите $\E(X)$, $\Var(X)$.
        \item {[5]} Найдите функцию распределения $F(x)$ величины $X$ и постройте её график.
    \end{enumerate}

    \begin{enumerate}
        \item $\E(X) = 1$; $\E(X^2) = 7/6$; $\Var(X) = 1/6$;
        \item     
        \[
            F(x) = \begin{cases}
                0, x < 0 \\
                x/4, x \in [0;1) \\
                x/4 + 1/2, x \in [1;2) \\
                1, x \geq 2
            \end{cases}
        \]
    
    \end{enumerate}
    




    \item {[10]} В лифт 12-этажного дома на первом этаже вошли 11 человек. 
    Каждый из них выходит независимо от других и равновероятно на любом из этажей, от второго до последнего. 
    \begin{enumerate}
        \item {[2]} Найдите вероятность того, что все выйдут на разных этажах.
        \item {[4]} Найдите вероятность того, что зашедшая в лифт Алиса выйдет на 6-м этаже или выше, если после 4-го этажа в лифте осталось пятеро.
        \item {[4]} Найдите вероятность того, что все пассажиры выйдут не выше 9-го этажа, если никто из них не вышел со 2-го по 6-й.
    \end{enumerate}

    \begin{enumerate}
        \item $11! / 11^11$;
        \item $5/11 \cdot 7/8 = 35/88$; Чтобы осознать, что события зависимы (!) достаточно представить ситуацию «после 4-го этажа в лифте осталось 11 человек».
        \item $(1/2)^{11}$;
    \end{enumerate}
\end{enumerate}


\end{document}

% здесь проектируемая часть


