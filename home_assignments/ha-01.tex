\documentclass[12pt]{article}

% \usepackage{physics}

\usepackage{hyperref}
\hypersetup{
    colorlinks=true,
    linkcolor=blue,
    filecolor=magenta,      
    urlcolor=cyan,
    pdftitle={Overleaf Example},
    pdfpagemode=FullScreen,
    }

\usepackage{tikzducks}

\usepackage{tikz} % картинки в tikz
\usepackage{microtype} % свешивание пунктуации

\usepackage{array} % для столбцов фиксированной ширины

\usepackage{indentfirst} % отступ в первом параграфе

\usepackage{sectsty} % для центрирования названий частей
\allsectionsfont{\centering}

\usepackage{amsmath, amsfonts, amssymb} % куча стандартных математических плюшек

\usepackage{comment}

\usepackage[top=2cm, left=1.2cm, right=1.2cm, bottom=2cm]{geometry} % размер текста на странице

\usepackage{lastpage} % чтобы узнать номер последней страницы

\usepackage{enumitem} % дополнительные плюшки для списков
%  например \begin{enumerate}[resume] позволяет продолжить нумерацию в новом списке
\usepackage{caption}

\usepackage{url} % to use \url{link to web}


\newcommand{\smallduck}{\begin{tikzpicture}[scale=0.3]
    \duck[
        cape=black,
        hat=black,
        mask=black
    ]
    \end{tikzpicture}}

\usepackage{fancyhdr} % весёлые колонтитулы
\pagestyle{fancy}
\lhead{}
\chead{}
\rhead{Домашние задания для самураев}
\lfoot{}
\cfoot{}
\rfoot{}

\renewcommand{\headrulewidth}{0.4pt}
\renewcommand{\footrulewidth}{0.4pt}

\usepackage{tcolorbox} % рамочки!

\usepackage{todonotes} % для вставки в документ заметок о том, что осталось сделать
% \todo{Здесь надо коэффициенты исправить}
% \missingfigure{Здесь будет Последний день Помпеи}
% \listoftodos - печатает все поставленные \todo'шки


% более красивые таблицы
\usepackage{booktabs}
% заповеди из докупентации:
% 1. Не используйте вертикальные линни
% 2. Не используйте двойные линии
% 3. Единицы измерения - в шапку таблицы
% 4. Не сокращайте .1 вместо 0.1
% 5. Повторяющееся значение повторяйте, а не говорите "то же"


\setcounter{MaxMatrixCols}{20}
% by crazy default pmatrix supports only 10 cols :)


\usepackage{fontspec}
\usepackage{libertine}
\usepackage{polyglossia}

\setmainlanguage{russian}
\setotherlanguages{english}

% download "Linux Libertine" fonts:
% http://www.linuxlibertine.org/index.php?id=91&L=1
% \setmainfont{Linux Libertine O} % or Helvetica, Arial, Cambria
% why do we need \newfontfamily:
% http://tex.stackexchange.com/questions/91507/
% \newfontfamily{\cyrillicfonttt}{Linux Libertine O}

\AddEnumerateCounter{\asbuk}{\russian@alph}{щ} % для списков с русскими буквами
\setlist[enumerate, 2]{label=\asbuk*),ref=\asbuk*}

%% эконометрические сокращения
\DeclareMathOperator{\Cov}{\mathbb{C}ov}
\DeclareMathOperator{\Corr}{\mathbb{C}orr}
\DeclareMathOperator{\Var}{\mathbb{V}ar}
\DeclareMathOperator{\col}{col}
\DeclareMathOperator{\row}{row}

\let\P\relax
\DeclareMathOperator{\P}{\mathbb{P}}

\DeclareMathOperator{\E}{\mathbb{E}}
% \DeclareMathOperator{\tr}{trace}
\DeclareMathOperator{\card}{card}

\DeclareMathOperator{\Convex}{Convex}

\newcommand \cN{\mathcal{N}}
\newcommand \dN{\mathcal{N}}
\newcommand \dBin{\mathrm{Bin}}


\newcommand \RR{\mathbb{R}}
\newcommand \NN{\mathbb{N}}





\begin{document}

\section*{Домашнее задание 1}

Дедлайн: 2024-09-24, 21:00.

\begin{enumerate}
\item Вася решает три задачи по теории вероятностей. 
Вероятности решить каждую задачу по отдельности равны $0.1$, $0.2$ и $0.5$. 
Решения задач никак не связаны между собой, знание ни одной из задач не помогает решить ни одну другую.
Обозначим буквой $N$ общее количество решенных задач. 

\begin{enumerate}
    \item Найдите все значения $N$ и их вероятности. 
    \item Найдите $\P(N \leq 2)$, $\E(N)$ и $\E(N^2)$.
\end{enumerate}

\item За работу Вася получает случайное целое количество $\xi$ баллов, равновероятно распределённое от $1$ до $n$. 

Найдите $\E(\xi)$, $\E(\xi^2)$, $\E(\xi^3)$.

\item Берём набор данных по ссылке 

\url{https://github.com/bdemeshev/hse_knad_probability_2024_2025/raw/main/home_assignments/ha01_data.csv}.

Здесь две переменных: $y_i$ — количество просмотренных Машей рилзов в день $i$ 
и бинарная переменная $x_i$ ($x_i = A$ — обычный день, $x_i = B$ — день дедлайна по теории вероятностей).

Рассмотрим две гипотезы. 
Нулевая гипотеза $H_0$: приближение дедлайна по вероятностям никак не влияет на количество просмотренных рилзов.
Альтернативная гипотеза $H_1$: приближение дедлайна в среднем снижает количество просмотренных рилзов. 

\begin{enumerate}
    \item Посчитайте фактическое значение статистики $S = \bar y_B - \bar y_A$.
    \item Предполагая, что $H_0$ верна, сгенерируйте $10000$ случайных перестановок меток $x$ и для 
    каждой перестановки посчитайте значение статистики $S^{\text{new}} = \bar y_B^{\text{new}} - \bar y_A^{\text{new}}$.
    \item Оцените $p$-значение, в данном случае $p$-значение — это вероятность $\P(S^{\text{new}} \leq S \mid S, H_0)$.
    \item Для принятия решения, отвергать или нет $H_0$, мы используем уровень значимости $\alpha = 0.05$.
    Отвергаем ли мы $H_0$?
\end{enumerate}

\end{enumerate}


\end{document}
