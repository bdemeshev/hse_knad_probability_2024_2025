\section*{Домашнее задание 6}

% knad, fall 2024

Дедлайн: 2024-11-01, 23:59.

\begin{enumerate}

\item Распределение вектора $(X, Y)$ задано таблицей

\begin{center}
    \begin{tabular}{lccc}
        \toprule
            & $Y = 1$  & $Y = 2$  & $Y = 3$ \\
        \midrule
        $X = 0$ & $0.2$  & $0.2$  & $0.1$ \\
        $X = 1$ & $0.5$  &  $0$   & $0$ \\
        \bottomrule
    \end{tabular}
\end{center}

\begin{enumerate}
    \item Найдите энтропии $\H(X)$, $\H(Y)$, $\H(X, Y)$.
    \item Найдите $\H(Y \mid X)$.
    \item Какое максимальное значение может принимать условная энтропия $\H(Y \mid X)$, 
    если $X$ принимает два значения, а $Y$ — три?
\end{enumerate}


\item Для дискретных величин $X$ и $Y$ докажите или опровергните утверждения:
\begin{enumerate}
\item $\H(X) + \H(Y \mid X) = \H(X, Y)$;
\item $\H(X, Y) \geq \H(X)$;
\item $\H(X^2) = \H(X)$;
\end{enumerate}

\item Время до прихода автобуса на остановку — неотрицательная случайная величина $X$ с функцией плотности. 
Андрей верит, что функция плотности $X$ имеет вид 
\[
a(x) = \begin{cases}
    \alpha \exp(-\alpha x), \text{ если } x \geq 0\\
    0, \text{ иначе}.
\end{cases}
\]
Борис верит, что функция плотности $X$ имеет вид 
\[
b(x) = \begin{cases}
    \beta \exp(-\beta x), \text{ если } x \geq 0\\
    0, \text{ иначе}.
\end{cases}
\]
Вова верит, что у $X$ есть какая-то функция плотности $c(x)$, а математическое ожидание такое, как думает Андрей. 
\begin{enumerate}
    \item Найдите математическое ожидание с точки зрения Андрея и Вовы, $\E_a(X)$.
    \item Найдите энтропию $\H(a)$. 
    \item Найдите кросс-энтропию $\CE(b || a)$. При каком $\alpha$ она минимальна?
    \item Найдите кросс-энтропию $\CE(c|| a)$.
    \item Чему равно максимальное значение энтропии $\H(c)$ и при какой функции плотности $c(x)$ достигается максимум?
\end{enumerate}





\end{enumerate}
