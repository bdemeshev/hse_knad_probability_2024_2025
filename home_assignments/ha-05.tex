
\documentclass[12pt]{article}

% \usepackage{physics}

\usepackage{hyperref}
\hypersetup{
    colorlinks=true,
    linkcolor=blue,
    filecolor=magenta,      
    urlcolor=cyan,
    pdftitle={Overleaf Example},
    pdfpagemode=FullScreen,
    }

\usepackage{tikzducks}

\usepackage{tikz} % картинки в tikz
\usepackage{microtype} % свешивание пунктуации

\usepackage{array} % для столбцов фиксированной ширины

\usepackage{indentfirst} % отступ в первом параграфе

\usepackage{sectsty} % для центрирования названий частей
\allsectionsfont{\centering}

\usepackage{amsmath, amsfonts, amssymb} % куча стандартных математических плюшек

\usepackage{comment}

\usepackage[top=2cm, left=1.2cm, right=1.2cm, bottom=2cm]{geometry} % размер текста на странице

\usepackage{lastpage} % чтобы узнать номер последней страницы

\usepackage{enumitem} % дополнительные плюшки для списков
%  например \begin{enumerate}[resume] позволяет продолжить нумерацию в новом списке
\usepackage{caption}

\usepackage{url} % to use \url{link to web}


\newcommand{\smallduck}{\begin{tikzpicture}[scale=0.3]
    \duck[
        cape=black,
        hat=black,
        mask=black
    ]
    \end{tikzpicture}}

\usepackage{fancyhdr} % весёлые колонтитулы
\pagestyle{fancy}
\lhead{}
\chead{}
\rhead{Домашние задания для самураев}
\lfoot{}
\cfoot{}
\rfoot{}

\renewcommand{\headrulewidth}{0.4pt}
\renewcommand{\footrulewidth}{0.4pt}

\usepackage{tcolorbox} % рамочки!

\usepackage{todonotes} % для вставки в документ заметок о том, что осталось сделать
% \todo{Здесь надо коэффициенты исправить}
% \missingfigure{Здесь будет Последний день Помпеи}
% \listoftodos - печатает все поставленные \todo'шки


% более красивые таблицы
\usepackage{booktabs}
% заповеди из докупентации:
% 1. Не используйте вертикальные линни
% 2. Не используйте двойные линии
% 3. Единицы измерения - в шапку таблицы
% 4. Не сокращайте .1 вместо 0.1
% 5. Повторяющееся значение повторяйте, а не говорите "то же"


\setcounter{MaxMatrixCols}{20}
% by crazy default pmatrix supports only 10 cols :)


\usepackage{fontspec}
\usepackage{libertine}
\usepackage{polyglossia}

\setmainlanguage{russian}
\setotherlanguages{english}

% download "Linux Libertine" fonts:
% http://www.linuxlibertine.org/index.php?id=91&L=1
% \setmainfont{Linux Libertine O} % or Helvetica, Arial, Cambria
% why do we need \newfontfamily:
% http://tex.stackexchange.com/questions/91507/
% \newfontfamily{\cyrillicfonttt}{Linux Libertine O}

\AddEnumerateCounter{\asbuk}{\russian@alph}{щ} % для списков с русскими буквами
\setlist[enumerate, 2]{label=\asbuk*),ref=\asbuk*}

%% эконометрические сокращения
\DeclareMathOperator{\pCorr}{\mathrm{pCorr}}
\DeclareMathOperator{\Cov}{\mathbb{C}ov}
\DeclareMathOperator{\Corr}{\mathbb{C}orr}
\DeclareMathOperator{\Var}{\mathbb{V}ar}
\DeclareMathOperator{\col}{col}
\DeclareMathOperator{\row}{row}

\let\P\relax
\DeclareMathOperator{\P}{\mathbb{P}}

\DeclareMathOperator{\E}{\mathbb{E}}
% \DeclareMathOperator{\tr}{trace}
\DeclareMathOperator{\card}{card}

\DeclareMathOperator{\Convex}{Convex}

\newcommand \cN{\mathcal{N}}
\newcommand \dN{\mathcal{N}}
\newcommand \dBin{\mathrm{Bin}}


\newcommand \RR{\mathbb{R}}
\newcommand \NN{\mathbb{N}}





\begin{document}

\section*{Домашнее задание 5}

Дедлайн: 2024-10-25, 23:59.

\begin{enumerate}

    \item Случайная величина $X$ распределена равномерно на $[0; 10]$, $Y = X^2$.
    \begin{enumerate}
        \item Найдите дисперсию $\Var(Y)$, стандартное отклонение $\sigma_Y$.
        \item Найдите $\Cov(X, Y)$, $\Corr(X, Y)$.
        \item Найдите $\Cov(6X + 2Y + 7, -2Y + 15)$, $\Corr(5 - 6X, 8 + 9Y)$, $\Var(2Y + 7)$.
        \item Предложите любую неслучайную функцию $h$ такую, что $\Corr(h(X), X) = 0$, $\Var(h(X)) > 0$.
    \end{enumerate}
    
    
    \item Назовём наилучшей линейной аппроксимацией величины $Y$ с помощью величины $X$ функцию вида $\hat Y = \alpha + \beta X$,
    где $\alpha$ и $\beta$ — константы, при которых величина $\E((Y - \hat Y)^2)$ минимальна.
    \begin{enumerate}
        \item Известно, что $\Cov(X, Y) = 10$, $\Var(X) = 40$, найдите $\beta$.
        \item Дополнительно известно, что $\E(Y) = 10$, $\E(X) = 80$, найдите $\alpha$.
    \end{enumerate}
    Допустим, что $a + b R$ — наилучшая линейная аппроксимация $L$ с помощью $R$, 
        а $c + d L$ — наилучшая линейная аппроксимация $R$ с помощью $L$. 
    \begin{enumerate}[resume]
        \item Выразите произведение $bd$ через корреляцию $\Corr(R, L)$.
    \end{enumerate}
        
    
    \item В анализе временных рядов иногда используют концепцию частной корреляции. 
    Частная корреляция между величинами $X$ и $Y$, очищенными от связи с величиной $W$, 
    равна обычной корреляции между величинами $X^* = X - \alpha W$ и $Y^* = Y - \beta W$,
    где константы $\alpha$ и $\beta$ находятся из условия некоррелированности $X^*$ c $W$ и 
    некоррелированности $Y^*$ с $W$.
    \[
    \pCorr(X, Y ; W) = \Corr(X^*, Y^*), \text{ где }
    \begin{cases}
        X^* = X - \alpha W, \quad \Cov(X^*, W) = 0, \\
        Y^* = Y - \beta W,  \quad \Cov(Y^*, W) = 0.
    \end{cases}  
    \]
    Величины $Y_1$, $Y_2$, $Y_3$ независимы и равномерны на отрезке $[0; 1]$,  
    $S_3 = Y_1 + Y_2 + Y_3$. % $S_2 = Y_1 + Y_2$,
    
    Найдите $\Corr(Y_1, Y_2)$ и $\pCorr(Y_1, Y_2; S_3)$.
    

\end{enumerate}


\end{document}
