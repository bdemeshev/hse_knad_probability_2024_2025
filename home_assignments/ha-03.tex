\documentclass[12pt]{article}

% \usepackage{physics}

\usepackage{hyperref}
\hypersetup{
    colorlinks=true,
    linkcolor=blue,
    filecolor=magenta,      
    urlcolor=cyan,
    pdftitle={Overleaf Example},
    pdfpagemode=FullScreen,
    }

\usepackage{tikzducks}

\usepackage{tikz} % картинки в tikz
\usepackage{microtype} % свешивание пунктуации

\usepackage{array} % для столбцов фиксированной ширины

\usepackage{indentfirst} % отступ в первом параграфе

\usepackage{sectsty} % для центрирования названий частей
\allsectionsfont{\centering}

\usepackage{amsmath, amsfonts, amssymb} % куча стандартных математических плюшек

\usepackage{comment}

\usepackage[top=2cm, left=1.2cm, right=1.2cm, bottom=2cm]{geometry} % размер текста на странице

\usepackage{lastpage} % чтобы узнать номер последней страницы

\usepackage{enumitem} % дополнительные плюшки для списков
%  например \begin{enumerate}[resume] позволяет продолжить нумерацию в новом списке
\usepackage{caption}

\usepackage{url} % to use \url{link to web}


\newcommand{\smallduck}{\begin{tikzpicture}[scale=0.3]
    \duck[
        cape=black,
        hat=black,
        mask=black
    ]
    \end{tikzpicture}}

\usepackage{fancyhdr} % весёлые колонтитулы
\pagestyle{fancy}
\lhead{}
\chead{}
\rhead{Домашние задания для самураев}
\lfoot{}
\cfoot{}
\rfoot{}

\renewcommand{\headrulewidth}{0.4pt}
\renewcommand{\footrulewidth}{0.4pt}

\usepackage{tcolorbox} % рамочки!

\usepackage{todonotes} % для вставки в документ заметок о том, что осталось сделать
% \todo{Здесь надо коэффициенты исправить}
% \missingfigure{Здесь будет Последний день Помпеи}
% \listoftodos - печатает все поставленные \todo'шки


% более красивые таблицы
\usepackage{booktabs}
% заповеди из докупентации:
% 1. Не используйте вертикальные линни
% 2. Не используйте двойные линии
% 3. Единицы измерения - в шапку таблицы
% 4. Не сокращайте .1 вместо 0.1
% 5. Повторяющееся значение повторяйте, а не говорите "то же"


\setcounter{MaxMatrixCols}{20}
% by crazy default pmatrix supports only 10 cols :)


\usepackage{fontspec}
\usepackage{libertine}
\usepackage{polyglossia}

\setmainlanguage{russian}
\setotherlanguages{english}

% download "Linux Libertine" fonts:
% http://www.linuxlibertine.org/index.php?id=91&L=1
% \setmainfont{Linux Libertine O} % or Helvetica, Arial, Cambria
% why do we need \newfontfamily:
% http://tex.stackexchange.com/questions/91507/
% \newfontfamily{\cyrillicfonttt}{Linux Libertine O}

\AddEnumerateCounter{\asbuk}{\russian@alph}{щ} % для списков с русскими буквами
\setlist[enumerate, 2]{label=\asbuk*),ref=\asbuk*}

%% эконометрические сокращения
\DeclareMathOperator{\Cov}{\mathbb{C}ov}
\DeclareMathOperator{\Corr}{\mathbb{C}orr}
\DeclareMathOperator{\Var}{\mathbb{V}ar}
\DeclareMathOperator{\col}{col}
\DeclareMathOperator{\row}{row}

\let\P\relax
\DeclareMathOperator{\P}{\mathbb{P}}

\DeclareMathOperator{\E}{\mathbb{E}}
% \DeclareMathOperator{\tr}{trace}
\DeclareMathOperator{\card}{card}

\DeclareMathOperator{\Convex}{Convex}

\newcommand \cN{\mathcal{N}}
\newcommand \dN{\mathcal{N}}
\newcommand \dBin{\mathrm{Bin}}


\newcommand \RR{\mathbb{R}}
\newcommand \NN{\mathbb{N}}





\begin{document}

\section*{Домашнее задание 3}

Дедлайн: 2024-10-08, 21:00.

\begin{enumerate}


\item В анкету включён вопрос, на который респонденты стесняются отвечать правдиво. 
Например, «Берёте ли Вы взятки?» или «Употребляете ли Вы наркотики?»
Чтобы стимулировать респондентов отвечать правдиво, используют следующий прием. 
Перед ответом на вопрос респондент в тайне от анкетирующего подкидывает один раз специальную монетку.
На аверсе монетки написано «Да = А, Нет = Б», на реверсе — «Да = Б, Нет = А». 
Ответ «Да» на нескромный вопрос является верным для доли $p$ всех людей. 

Монетка неправильная и выпадает стороной «Да= А, Нет = Б» с вероятностью $0.6$.

\begin{enumerate}
    \item Какова вероятность того, что ответ «Да» для данного индивида верен, если он написал «A» и следовал указаниям монетки?
    \item Какова вероятность того, что ответ «Да» для данного индивида верен, если он подбрасывал специальную монету 3 раза,
    следовал каждый раз предлагаемой кодировке и написал «А», «Б», «А»?
\end{enumerate}

\item Илон Маск изобрёл новый кубик под названием Model-6. 
Он взял правильный игральный кубик и правильный кубик с неподписанными гранями. 
Он подкинул шесть раз правильный игральный кубик и заполнил по очереди все грани изначально чистого кубика результатами бросков правильного кубика.
\begin{enumerate}
    \item Какова вероятность того, что в первом броске Model-6 выпало 6, если во втором броске Model-6 выпало 6?
    \item Зависимы ли результаты бросков Model-6?
    \item Чему равно ожидаемое количество шестёрок, выпавших в процессе изготовления Model-6, если при шести бросках Model-6 выпало три шестёрки?
\end{enumerate}


\item Алиса и Боб снова подкидывают монетку неограниченное число раз. 
Монетка выпадает решкой $H$ и орлом $T$ равновероятно. 
Алиса выигрывает, если последовательность $HHT$ выпадет раньше, а Боб — если раньше выпадет $HTH$.

Рассмотрим множество исходов этого эксперимента $\Omega = \{HHT, HTH, HHHT, THTH, THHT, \dots \}$
и производящую функцию исходов $f(H, T) = HHT + HTH + HHHT + THTH + THHT + \dots$.
Здесь аргументы $H$ и $T$ некоммутативны. 
Обозначим $X$ — количество решек $H$, $Y$ — количество орлов $T$.

\begin{enumerate}
    \item Укажите, как с помощью производных и подстановок раздобыть из функции $f(H, T)$ величины $\P(X = 10)$,
    $\P(X = 5, Y=5)$, $\E(X)$, $\E(X^3)$, $\E(X^2Y^3)$.
    \item С помощью метода первого шага составьте систему линейных уравнений, из которой можно найти $f(H, T)$. 
    \item Решите эту систему, предполагая коммутативность $H$ и $T$. 
    \item Завершите вычисление $\P(X = 10)$, $\P(X = 5, Y=5)$, $\E(X)$, $\E(X^3)$, $\E(X^2Y^3)$.
\end{enumerate}

Явное уточнение: конечно, в этой задаче можно использовать \verb|sympy| или другой пакет для символьного решения системы или вычисления производных. 

\end{enumerate}


\end{document}
