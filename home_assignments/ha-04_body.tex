\section*{Домашнее задание 4}


У этого задания нет дедлайна и за него нет оценки. 
Если очень хочется что-то куда-то загрузить, то можно отправить своему семинаристу мемасик по теории вероятностей :)


\begin{enumerate}
\item Случайная величина $X$ принимает значения $1$, $2$, $3$ и $4$ с вероятностями $0.1$, $0.2$, $0.3$, $0.4$.
\begin{enumerate}
    \item Нарисуйте функцию распределения величины $X$, $F_X(x)$.
    \item Какой вероятностный смысл имеет площадь над функцией распределения $F_X(x)$ на участке $x \in [0;\infty)$?
    \item Нарисуйте функцию распределения случайной величины $Y = F_X(X)$.
\end{enumerate}

\item Функция плотности случайной величины $Y$ равна $c y^2$ на отрезке $[0;2]$ и нулю иначе. 
\begin{enumerate}
    \item Найдите константу $c$. 
    \item Найдите функцию распределения $Y$.
    \item Найдите $\P(Y > 1)$, $\P(Y = 0.75)$, $\E(Y)$, $\E(Y^2)$.
    \item Найдите функцию производящую моменты $Y$, $m_Y(t)$.
    \item Найдите $\P(Y > 1.5 \mid Y > 1)$, $\E(Y \mid Y>1)$, $\E(Y^2 \mid Y > 1)$.
    \item Найдите функцию плотности величины $W = 1 / Y$.
\end{enumerate}

\item Случайная велина $U$ равномерна на отрезке $[0;10]$, $Y = \min \{U^2, 25\}$.
\begin{enumerate}
    \item Запишите вероятность $\P(Y \in [y; y + \Delta])$ с точностью до $o(\Delta)$.
    \item Найдите функцию распределения $Y$.
    \item Найдите $\P(Y > 10)$, $\E(Y)$, $\E(Y^2)$.
    \item Найдите $\P(Y > 10 \mid Y > 5)$, $\E(Y \mid Y>5)$, $\E(Y^2 \mid Y > 5)$.
\end{enumerate}
\end{enumerate}
