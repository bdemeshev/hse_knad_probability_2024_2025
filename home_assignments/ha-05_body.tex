\section*{Домашнее задание 5}

Дедлайн: 2024-10-23, 21:00.

\begin{enumerate}

    \item Случайная величина $X$ распределена равномерно на $[0; 10]$, $Y = X^2$.
    \begin{enumerate}
        \item Найдите дисперсию $\Var(Y)$, стандартное отклонение $\sigma_Y$.
        \item Найдите $\Cov(X, Y)$, $\Corr(X, Y)$.
        \item Найдите $\Cov(6X + 2Y + 7, -2Y + 15)$, $\Corr(5 - 6X, 8 + 9Y)$, $\Var(2Y + 7)$.
        \item Предложите любую неслучайную функцию $h$ такую, что $\Corr(h(X), X) = 0$, $\Var(h(X)) > 0$.
    \end{enumerate}
    
    
    \item Назовём наилучшей линейной аппроксимацией величины $Y$ с помощью величины $X$ функцию вида $\hat Y = \alpha + \beta X$,
    где $\alpha$ и $\beta$ — константы, при которых величина $\E((Y - \hat Y)^2)$ минимальна.
    \begin{enumerate}
        \item Известно, что $\Cov(X, Y) = 10$, $\Var(X) = 40$, найдите $\beta$.
        \item Дополнительно известно, что $\E(Y) = 10$, $\E(X) = 80$, найдите $\alpha$.
    \end{enumerate}
    Допустим, что $a + b R$ — наилучшая линейная аппроксимация $L$ с помощью $R$, 
        а $c + d L$ — наилучшая линейная аппроксимация $R$ с помощью $L$. 
    \begin{enumerate}[resume]
        \item Выразите произведение $bd$ через корреляцию $\Corr(R, L)$.
    \end{enumerate}
        
    
    \item В анализе временных рядов иногда используют концепцию частной корреляции. 
    Частная корреляция между величинами $X$ и $Y$, очищенными от связи с величиной $W$, 
    равна обычной корреляции между величинами $X^* = X - \alpha W$ и $Y^* = Y - \beta W$,
    где константы $\alpha$ и $\beta$ находятся из условия некоррелированности $X^*$ c $W$ и 
    некоррелированности $Y^*$ с $W$.
    \[
    \pCorr(X, Y ; W) = \Corr(X^*, Y^*), \text{ где }
    \begin{cases}
        X^* = X - \alpha W, \quad \Cov(X^*, W) = 0, \\
        Y^* = Y - \beta W,  \quad \Cov(Y^*, W) = 0.
    \end{cases}  
    \]
    Величины $Y_1$, $Y_2$, $Y_3$ независимы и равномерны на отрезке $[0; 1]$,  
    $S_3 = Y_1 + Y_2 + Y_3$. % $S_2 = Y_1 + Y_2$,
    
    Найдите $\Corr(Y_1, Y_2)$ и $\pCorr(Y_1, Y_2; S_3)$.
    

\end{enumerate}
