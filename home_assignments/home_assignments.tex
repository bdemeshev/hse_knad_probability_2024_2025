% arara: xelatex
\documentclass[12pt]{article}

% \usepackage{physics}


\usepackage{hyperref}
\hypersetup{
    colorlinks=true,
    linkcolor=blue,
    filecolor=magenta,      
    urlcolor=cyan,
    pdftitle={Overleaf Example},
    pdfpagemode=FullScreen,
    }

\usepackage{tikzducks}

\usepackage{tikz} % картинки в tikz
\usepackage{microtype} % свешивание пунктуации

\usepackage{array} % для столбцов фиксированной ширины

\usepackage{indentfirst} % отступ в первом параграфе

\usepackage{sectsty} % для центрирования названий частей
\allsectionsfont{\centering}

\usepackage{amsmath, amsfonts, amssymb} % куча стандартных математических плюшек

\usepackage{comment}

\usepackage[top=2cm, left=1.2cm, right=1.2cm, bottom=2cm]{geometry} % размер текста на странице

\usepackage{lastpage} % чтобы узнать номер последней страницы

\usepackage{enumitem} % дополнительные плюшки для списков
%  например \begin{enumerate}[resume] позволяет продолжить нумерацию в новом списке
\usepackage{caption}

\usepackage{url} % to use \url{link to web}


\newcommand{\smallduck}{\begin{tikzpicture}[scale=0.3]
    \duck[
        cape=black,
        hat=black,
        mask=black
    ]
    \end{tikzpicture}}

\usepackage{fancyhdr} % весёлые колонтитулы
\pagestyle{fancy}
\lhead{Теория вероятностей и статистика: кнад}
\chead{}
\rhead{Домашние задания для самураев}
\lfoot{}
\cfoot{}
\rfoot{}

\renewcommand{\headrulewidth}{0.4pt}
\renewcommand{\footrulewidth}{0.4pt}

\usepackage{tcolorbox} % рамочки!

\usepackage{todonotes} % для вставки в документ заметок о том, что осталось сделать
% \todo{Здесь надо коэффициенты исправить}
% \missingfigure{Здесь будет Последний день Помпеи}
% \listoftodos - печатает все поставленные \todo'шки


% более красивые таблицы
\usepackage{booktabs}
% заповеди из докупентации:
% 1. Не используйте вертикальные линни
% 2. Не используйте двойные линии
% 3. Единицы измерения - в шапку таблицы
% 4. Не сокращайте .1 вместо 0.1
% 5. Повторяющееся значение повторяйте, а не говорите "то же"


\setcounter{MaxMatrixCols}{20}
% by crazy default pmatrix supports only 10 cols :)


\usepackage{fontspec}
\usepackage{libertine}
\usepackage{polyglossia}

\setmainlanguage{russian}
\setotherlanguages{english}

% download "Linux Libertine" fonts:
% http://www.linuxlibertine.org/index.php?id=91&L=1
% \setmainfont{Linux Libertine O} % or Helvetica, Arial, Cambria
% why do we need \newfontfamily:
% http://tex.stackexchange.com/questions/91507/
% \newfontfamily{\cyrillicfonttt}{Linux Libertine O}

\AddEnumerateCounter{\asbuk}{\russian@alph}{щ} % для списков с русскими буквами
\setlist[enumerate, 2]{label=\asbuk*),ref=\asbuk*}

%% эконометрические сокращения
\DeclareMathOperator{\Cov}{\mathbb{C}ov}
\DeclareMathOperator{\pCorr}{\mathrm{pCorr}}
\DeclareMathOperator{\Corr}{\mathbb{C}orr}
\DeclareMathOperator{\Var}{\mathbb{V}ar}
\DeclareMathOperator{\col}{col}
\DeclareMathOperator{\row}{row}

\let\P\relax
\DeclareMathOperator{\P}{\mathbb{P}}

\let\H\relax
\DeclareMathOperator{\H}{\mathbb{H}}

\DeclareMathOperator{\CE}{\mathrm{CE}}



\DeclareMathOperator{\E}{\mathbb{E}}
% \DeclareMathOperator{\tr}{trace}
\DeclareMathOperator{\card}{card}

\DeclareMathOperator{\Convex}{Convex}

\newcommand \cN{\mathcal{N}}
\newcommand \RR{\mathbb{R}}
\newcommand \NN{\mathbb{N}}





\begin{document}


\section*{Домашнее задание 1}

Дедлайн: 2024-09-24, 21:00.

\begin{enumerate}
\item Вася решает три задачи по теории вероятностей. 
Вероятности решить каждую задачу по отдельности равны $0.1$, $0.2$ и $0.5$. 
Решения задач никак не связаны между собой, знание ни одной из задач не помогает решить ни одну другую.
Обозначим буквой $N$ общее количество решенных задач. 

\begin{enumerate}
    \item Найдите все значения $N$ и их вероятности. 
    \item Найдите $\P(N \leq 2)$, $\E(N)$ и $\E(N^2)$.
\end{enumerate}

\item За работу Вася получает случайное целое количество $\xi$ баллов, равновероятно распределённое от $1$ до $n$. 

Найдите $\E(\xi)$, $\E(\xi^2)$, $\E(\xi^3)$.

\item Берём набор данных по ссылке 

\url{https://github.com/bdemeshev/hse_knad_probability_2024_2025/raw/main/home_assignments/ha01_data.csv}.

Здесь две переменных: $y_i$ — количество просмотренных Машей рилзов в день $i$ 
и бинарная переменная $x_i$ ($x_i = A$ — обычный день, $x_i = B$ — день дедлайна по теории вероятностей).

Рассмотрим две гипотезы. 
Нулевая гипотеза $H_0$: приближение дедлайна по вероятностям никак не влияет на количество просмотренных рилзов.
Альтернативная гипотеза $H_1$: приближение дедлайна в среднем снижает количество просмотренных рилзов. 

\begin{enumerate}
    \item Посчитайте фактическое значение статистики $S = \bar y_B - \bar y_A$.
    \item Предполагая, что $H_0$ верна, сгенерируйте $10000$ случайных перестановок меток $x$ и для 
    каждой перестановки посчитайте значение статистики $S^{\text{new}} = \bar y_B^{\text{new}} - \bar y_A^{\text{new}}$.
    \item Оцените $p$-значение, в данном случае $p$-значение — это вероятность $\P(S^{\text{new}} \leq S \mid S, H_0)$.
    \item Для принятия решения, отвергать или нет $H_0$, мы используем уровень значимости $\alpha = 0.05$.
    Отвергаем ли мы $H_0$?
\end{enumerate}

\end{enumerate}


\section*{Домашнее задание 2}

Дедлайн: 2024-10-01, 21:00.

\begin{enumerate}
\item Монетка выпадает орлом $T$ с вероятностью $0.2$ и решкой $H$ — с вероятностью $0.8$.
Илон Маск подбрасывает её $100$ раз. 
За каждую выпавшую комбинацию $THT$ он получает $1\$$, а за каждую комбинацию $HHHHHH$ — платит $1\$$.

Чему равен ожидаемый выигрыш Маска в эту игру?

Уточнение: комбинации могут пересекаться, например, за $THTHT$ Маск получит $2\$$.

\item Бармен Огненной Зебры разбавляет каждую кружку пива независимо от других с общеизвестной вероятностью $p \in (0;1)$.
Ковбой Джо заходит в бар и первым делом сразу заказывает три кружки пива и выпивает их.
Затем Джо заказывает по две кружки пива за один раз. 

После 3-й, 5-й, 7-й, 9-й, 11-й и далее через каждые две кружки Джо прислушивается к своим ощущениям.
Если не менее двух кружек пива из последних трёх кружек разбавлены, то Джо разносит бар к чертям собачьим. 

\begin{enumerate}
    \item Сколько кружек пива в среднем успеет выпить Джо прежде чем разнесёт Огненную Зебру?
    \item Если все три последние кружки пива разбавлены, то Джо разносит не только Огненную Зебру, 
    но и всю прилежащую улицу. Какова вероятность данного сценария?
\end{enumerate}


\item Камала Харрис подбрасывает кубик до первого выпадения восьмёрки.
Все грани кубика выпадают равновероятно, однако на его шести гранях написаны числа 3, 4, 5, 6, 7, 8.
Дональд Трамп подбрасывает правильный октаэдр до выпадения восьмёрки. 
На гранях октаэдра написаны числа от 1, 2, 3, 4, 5, 6, 7, 8.

\begin{enumerate}
    \item Постройте гистограмму числа бросков кубика по $B = 10000$ экспериментов. 
    \item Оцените безусловное математическое ожидание числа бросков кубика.
    \item Оцените безусловную вероятность окончания игры быстрее, чем за 5 бросков кубика.
    \item Постройте условную гистограмму числа бросков октаэдра, если известно что грани 1 и 2 не выпадали. 
    Общее количество экспериментов здесь должно быть таким, чтобы число число экспериментов, где не выпадали грани 1 и 2 оказалось равным $B = 10000$.
    \item Оцените условное ожидание числа бросков октаэдра, если грани 1 и 2 не выпадали. 
    \item Оцените условную вероятность окончания игры быстрее, чем за 5 бросков октаэдра, если грани 1 и 2 не выпадали.
\end{enumerate}

\end{enumerate}


\section*{Домашнее задание 3}

Дедлайн: 2024-10-08, 21:00.

\begin{enumerate}


\item В анкету включён вопрос, на который респонденты стесняются отвечать правдиво. 
Например, «Берёте ли Вы взятки?» или «Употребляете ли Вы наркотики?»
Чтобы стимулировать респондентов отвечать правдиво, используют следующий прием. 
Перед ответом на вопрос респондент в тайне от анкетирующего подкидывает один раз специальную монетку.
На аверсе монетки написано «Да = А, Нет = Б», на реверсе — «Да = Б, Нет = А». 
Ответ «Да» на нескромный вопрос является верным для доли $p$ всех людей. 

Монетка неправильная и выпадает стороной «Да= А, Нет = Б» с вероятностью $0.6$.

\begin{enumerate}
    \item Какова вероятность того, что ответ «Да» для данного индивида верен, если он написал «A» и следовал указаниям монетки?
    \item Какова вероятность того, что ответ «Да» для данного индивида верен, если он подбрасывал специальную монету 3 раза,
    следовал каждый раз предлагаемой кодировке и написал «А», «Б», «А»?
\end{enumerate}

\item Илон Маск изобрёл новый кубик под названием Model-6. 
Он взял правильный игральный кубик и правильный кубик с неподписанными гранями. 
Он подкинул шесть раз правильный игральный кубик и заполнил по очереди все грани изначально чистого кубика результатами бросков правильного кубика.
\begin{enumerate}
    \item Какова вероятность того, что в первом броске Model-6 выпало 6, если во втором броске Model-6 выпало 6?
    \item Зависимы ли результаты бросков Model-6?
    \item Чему равно ожидаемое количество шестёрок, выпавших в процессе изготовления Model-6, если при шести бросках Model-6 выпало три шестёрки?
\end{enumerate}


\item Алиса и Боб снова подкидывают монетку неограниченное число раз. 
Монетка выпадает решкой $H$ и орлом $T$ равновероятно. 
Алиса выигрывает, если последовательность $HHT$ выпадет раньше, а Боб — если раньше выпадет $HTH$.

Рассмотрим множество исходов этого эксперимента $\Omega = \{HHT, HTH, HHHT, THTH, THHT, \dots \}$
и производящую функцию исходов $f(H, T) = HHT + HTH + HHHT + THTH + THHT + \dots$.
Здесь аргументы $H$ и $T$ некоммутативны. 
Обозначим $X$ — количество решек $H$, $Y$ — количество орлов $T$.

\begin{enumerate}
    \item Укажите, как с помощью производных и подстановок раздобыть из функции $f(H, T)$ величины $\P(X = 10)$,
    $\P(X = 5, Y=5)$, $\E(X)$, $\E(X^3)$, $\E(X^2Y^3)$.
    \item С помощью метода первого шага составьте систему линейных уравнений, из которой можно найти $f(H, T)$. 
    \item Решите эту систему, предполагая коммутативность $H$ и $T$. 
    \item Завершите вычисление $\P(X = 10)$, $\P(X = 5, Y=5)$, $\E(X)$, $\E(X^3)$, $\E(X^2Y^3)$.
\end{enumerate}

Явное уточнение: конечно, в этой задаче можно использовать \verb|sympy| или другой пакет для символьного решения системы или вычисления производных. 

\end{enumerate}


\section*{Домашнее задание 4}


У этого задания нет дедлайна и за него нет оценки. 
Если очень хочется что-то куда-то загрузить, то можно отправить своему семинаристу мемасик по теории вероятностей :)


\begin{enumerate}
\item Случайная величина $X$ принимает значения $1$, $2$, $3$ и $4$ с вероятностями $0.1$, $0.2$, $0.3$, $0.4$.
\begin{enumerate}
    \item Нарисуйте функцию распределения величины $X$, $F_X(x)$.
    \item Какой вероятностный смысл имеет площадь над функцией распределения $F_X(x)$ на участке $x \in [0;\infty)$?
    \item Нарисуйте функцию распределения случайной величины $Y = F_X(X)$.
\end{enumerate}

\item Функция плотности случайной величины $Y$ равна $c y^2$ на отрезке $[0;2]$ и нулю иначе. 
\begin{enumerate}
    \item Найдите константу $c$. 
    \item Найдите функцию распределения $Y$.
    \item Найдите $\P(Y > 1)$, $\P(Y = 0.75)$, $\E(Y)$, $\E(Y^2)$.
    \item Найдите функцию производящую моменты $Y$, $m_Y(t)$.
    \item Найдите $\P(Y > 1.5 \mid Y > 1)$, $\E(Y \mid Y>1)$, $\E(Y^2 \mid Y > 1)$.
    \item Найдите функцию плотности величины $W = 1 / Y$.
\end{enumerate}

\item Случайная велина $U$ равномерна на отрезке $[0;10]$, $Y = \min \{U^2, 25\}$.
\begin{enumerate}
    \item Запишите вероятность $\P(Y \in [y; y + \Delta])$ с точностью до $o(\Delta)$.
    \item Найдите функцию распределения $Y$.
    \item Найдите $\P(Y > 10)$, $\E(Y)$, $\E(Y^2)$.
    \item Найдите $\P(Y > 10 \mid Y > 5)$, $\E(Y \mid Y>5)$, $\E(Y^2 \mid Y > 5)$.
\end{enumerate}
\end{enumerate}


\section*{Домашнее задание 5}

Дедлайн: 2024-10-25, 23:59.

\begin{enumerate}

    \item Случайная величина $X$ распределена равномерно на $[0; 10]$, $Y = X^2$.
    \begin{enumerate}
        \item Найдите дисперсию $\Var(Y)$, стандартное отклонение $\sigma_Y$.
        \item Найдите $\Cov(X, Y)$, $\Corr(X, Y)$.
        \item Найдите $\Cov(6X + 2Y + 7, -2Y + 15)$, $\Corr(5 - 6X, 8 + 9Y)$, $\Var(2Y + 7)$.
        \item Предложите любую неслучайную функцию $h$ такую, что $\Corr(h(X), X) = 0$, $\Var(h(X)) > 0$.
    \end{enumerate}
    
    
    \item Назовём наилучшей линейной аппроксимацией величины $Y$ с помощью величины $X$ функцию вида $\hat Y = \alpha + \beta X$,
    где $\alpha$ и $\beta$ — константы, при которых величина $\E((Y - \hat Y)^2)$ минимальна.
    \begin{enumerate}
        \item Известно, что $\Cov(X, Y) = 10$, $\Var(X) = 40$, найдите $\beta$.
        \item Дополнительно известно, что $\E(Y) = 10$, $\E(X) = 80$, найдите $\alpha$.
    \end{enumerate}
    Допустим, что $a + b R$ — наилучшая линейная аппроксимация $L$ с помощью $R$, 
        а $c + d L$ — наилучшая линейная аппроксимация $R$ с помощью $L$. 
    \begin{enumerate}[resume]
        \item Выразите произведение $bd$ через корреляцию $\Corr(R, L)$.
    \end{enumerate}
        
    
    \item В анализе временных рядов иногда используют концепцию частной корреляции. 
    Частная корреляция между величинами $X$ и $Y$, очищенными от связи с величиной $W$, 
    равна обычной корреляции между величинами $X^* = X - \alpha W$ и $Y^* = Y - \beta W$,
    где константы $\alpha$ и $\beta$ находятся из условия некоррелированности $X^*$ c $W$ и 
    некоррелированности $Y^*$ с $W$.
    \[
    \pCorr(X, Y ; W) = \Corr(X^*, Y^*), \text{ где }
    \begin{cases}
        X^* = X - \alpha W, \quad \Cov(X^*, W) = 0, \\
        Y^* = Y - \beta W,  \quad \Cov(Y^*, W) = 0.
    \end{cases}  
    \]
    Величины $Y_1$, $Y_2$, $Y_3$ независимы и равномерны на отрезке $[0; 1]$,  
    $S_3 = Y_1 + Y_2 + Y_3$. % $S_2 = Y_1 + Y_2$,
    
    Найдите $\Corr(Y_1, Y_2)$ и $\pCorr(Y_1, Y_2; S_3)$.
    

\end{enumerate}


\section*{Домашнее задание 6}

% knad, fall 2024

Дедлайн: 2024-11-01, 23:59.

\begin{enumerate}

\item Распределение вектора $(X, Y)$ задано таблицей

\begin{center}
    \begin{tabular}{lccc}
        \toprule
            & $Y = 1$  & $Y = 2$  & $Y = 3$ \\
        \midrule
        $X = 0$ & $0.2$  & $0.2$  & $0.1$ \\
        $X = 1$ & $0.5$  &  $0$   & $0$ \\
        \bottomrule
    \end{tabular}
\end{center}

\begin{enumerate}
    \item Найдите энтропии $\H(X)$, $\H(Y)$, $\H(X, Y)$.
    \item Найдите $\H(Y \mid X)$.
    \item Какое максимальное значение может принимать условная энтропия $\H(Y \mid X)$, 
    если $X$ принимает два значения, а $Y$ — три?
\end{enumerate}


\item Для дискретных величин $X$ и $Y$ докажите или опровергните утверждения:
\begin{enumerate}
\item $\H(X) + \H(Y \mid X) = \H(X, Y)$;
\item $\H(X, Y) \geq \H(X)$;
\item $\H(X^2) = \H(X)$;
\end{enumerate}

\item Время до прихода автобуса на остановку — неотрицательная случайная величина $X$ с функцией плотности. 
Андрей верит, что функция плотности $X$ имеет вид 
\[
a(x) = \begin{cases}
    \alpha \exp(-\alpha x), \text{ если } x \geq 0\\
    0, \text{ иначе}.
\end{cases}
\]
Борис верит, что функция плотности $X$ имеет вид 
\[
b(x) = \begin{cases}
    \beta \exp(-\beta x), \text{ если } x \geq 0\\
    0, \text{ иначе}.
\end{cases}
\]
Вова верит, что у $X$ есть какая-то функция плотности $c(x)$, а математическое ожидание такое, как думает Андрей. 
\begin{enumerate}
    \item Найдите математическое ожидание с точки зрения Андрея и Вовы, $\E_a(X)$.
    \item Найдите энтропию $\H(a)$. 
    \item Найдите кросс-энтропию $\CE(b || a)$. При каком $\alpha$ она минимальна?
    \item Найдите кросс-энтропию $\CE(c|| a)$.
    \item Чему равно максимальное значение энтропии $\H(c)$ и при какой функции плотности $c(x)$ достигается максимум?
\end{enumerate}





\end{enumerate}

\end{document}

% здесь проектируемая часть

\end{document}

