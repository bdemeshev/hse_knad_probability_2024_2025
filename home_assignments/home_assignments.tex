% arara: xelatex
\documentclass[12pt]{article}

% \usepackage{physics}


\usepackage{hyperref}
\hypersetup{
    colorlinks=true,
    linkcolor=blue,
    filecolor=magenta,      
    urlcolor=cyan,
    pdftitle={Overleaf Example},
    pdfpagemode=FullScreen,
    }

\usepackage{tikzducks}

\usepackage{tikz} % картинки в tikz
\usepackage{microtype} % свешивание пунктуации

\usepackage{array} % для столбцов фиксированной ширины

\usepackage{indentfirst} % отступ в первом параграфе

\usepackage{sectsty} % для центрирования названий частей
\allsectionsfont{\centering}

\usepackage{amsmath, amsfonts, amssymb} % куча стандартных математических плюшек

\usepackage{comment}

\usepackage[top=2cm, left=1.2cm, right=1.2cm, bottom=2cm]{geometry} % размер текста на странице

\usepackage{lastpage} % чтобы узнать номер последней страницы

\usepackage{enumitem} % дополнительные плюшки для списков
%  например \begin{enumerate}[resume] позволяет продолжить нумерацию в новом списке
\usepackage{caption}

\usepackage{url} % to use \url{link to web}


\newcommand{\smallduck}{\begin{tikzpicture}[scale=0.3]
    \duck[
        cape=black,
        hat=black,
        mask=black
    ]
    \end{tikzpicture}}

\usepackage{fancyhdr} % весёлые колонтитулы
\pagestyle{fancy}
\lhead{Теория вероятностей и статистика: кнад}
\chead{}
\rhead{Домашние задания для самураев}
\lfoot{}
\cfoot{}
\rfoot{}

\renewcommand{\headrulewidth}{0.4pt}
\renewcommand{\footrulewidth}{0.4pt}

\usepackage{tcolorbox} % рамочки!

\usepackage{todonotes} % для вставки в документ заметок о том, что осталось сделать
% \todo{Здесь надо коэффициенты исправить}
% \missingfigure{Здесь будет Последний день Помпеи}
% \listoftodos - печатает все поставленные \todo'шки


% более красивые таблицы
\usepackage{booktabs}
% заповеди из докупентации:
% 1. Не используйте вертикальные линни
% 2. Не используйте двойные линии
% 3. Единицы измерения - в шапку таблицы
% 4. Не сокращайте .1 вместо 0.1
% 5. Повторяющееся значение повторяйте, а не говорите "то же"


\setcounter{MaxMatrixCols}{20}
% by crazy default pmatrix supports only 10 cols :)


\usepackage{fontspec}
\usepackage{libertine}
\usepackage{polyglossia}

\setmainlanguage{russian}
\setotherlanguages{english}

% download "Linux Libertine" fonts:
% http://www.linuxlibertine.org/index.php?id=91&L=1
% \setmainfont{Linux Libertine O} % or Helvetica, Arial, Cambria
% why do we need \newfontfamily:
% http://tex.stackexchange.com/questions/91507/
% \newfontfamily{\cyrillicfonttt}{Linux Libertine O}

\AddEnumerateCounter{\asbuk}{\russian@alph}{щ} % для списков с русскими буквами
\setlist[enumerate, 2]{label=\asbuk*),ref=\asbuk*}

%% эконометрические сокращения
\DeclareMathOperator{\Cov}{\mathbb{C}ov}
\DeclareMathOperator{\Corr}{\mathbb{C}orr}
\DeclareMathOperator{\Var}{\mathbb{V}ar}
\DeclareMathOperator{\col}{col}
\DeclareMathOperator{\row}{row}

\let\P\relax
\DeclareMathOperator{\P}{\mathbb{P}}

\DeclareMathOperator{\E}{\mathbb{E}}
% \DeclareMathOperator{\tr}{trace}
\DeclareMathOperator{\card}{card}

\DeclareMathOperator{\Convex}{Convex}

\newcommand \cN{\mathcal{N}}
\newcommand \RR{\mathbb{R}}
\newcommand \NN{\mathbb{N}}





\begin{document}

\section*{Домашнее задание 1}

Дедлайн: 2024-09-24, 21:00.

\begin{enumerate}
\item Вася решает три задачи по теории вероятностей. 
Вероятности решить каждую задачу по отдельности равны $0.1$, $0.2$ и $0.5$. 
Решения задач никак не связаны между собой, знание ни одной из задач не помогает решить ни одну другую.
Обозначим буквой $N$ общее количество решенных задач. 

\begin{enumerate}
    \item Найдите все значения $N$ и их вероятности. 
    \item Найдите $\P(N \leq 2)$, $\E(N)$ и $\E(N^2)$.
\end{enumerate}

\item За работу Вася получает случайное целое количество $\xi$ баллов, равновероятно распределённое от $1$ до $n$. 

Найдите $\E(\xi)$, $\E(\xi^2)$, $\E(\xi^3)$.

\item Берём набор данных по ссылке 

\url{https://github.com/bdemeshev/hse_knad_probability_2024_2025/raw/main/home_assignments/ha01_data.csv}.

Здесь две переменных: $y_i$ — количество просмотренных Машей рилзов в день $i$ 
и бинарная переменная $x_i$ ($x_i = A$ — обычный день, $x_i = B$ — день дедлайна по теории вероятностей).

Рассмотрим две гипотезы. 
Нулевая гипотеза $H_0$: приближение дедлайна по вероятностям никак не влияет на количество просмотренных рилзов.
Альтернативная гипотеза $H_1$: приближение дедлайна в среднем снижает количество просмотренных рилзов. 

\begin{enumerate}
    \item Посчитайте фактическое значение статистики $S = \bar y_B - \bar y_A$.
    \item Предполагая, что $H_0$ верна, сгенерируйте $10000$ случайных перестановок меток $x$ и для 
    каждой перестановки посчитайте значение статистики $S^{\text{new}} = \bar y_B^{\text{new}} - \bar y_A^{\text{new}}$.
    \item Оцените $p$-значение, в данном случае $p$-значение — это вероятность $\P(S^{\text{new}} \leq S \mid S, H_0)$.
    \item Для принятия решения, отвергать или нет $H_0$, мы используем уровень значимости $\alpha = 0.05$.
    Отвергаем ли мы $H_0$?
\end{enumerate}

\end{enumerate}


\section*{Домашнее задание 2}

Дедлайн: 2024-10-01, 21:00.

\begin{enumerate}
\item Монетка выпадает орлом $T$ с вероятностью $0.2$ и решкой $H$ — с вероятностью $0.8$.
Илон Маск подбрасывает её $100$ раз. 
За каждую выпавшую комбинацию $THT$ он получает $1\$$, а за каждую комбинацию $HHHHHH$ — платит $1\$$.

Чему равен ожидаемый выигрыш Маска в эту игру?

Уточнение: комбинации могут пересекаться, например, за $THTHT$ Маск получит $2\$$.

\item Бармен Огненной Зебры разбавляет каждую кружку пива независимо от других с общеизвестной вероятностью $p \in (0;1)$.
Ковбой Джо заходит в бар и первым делом сразу заказывает три кружки пива и выпивает их.
Затем Джо заказывает по две кружки пива за один раз. 

После 3-й, 5-й, 7-й, 9-й, 11-й и далее через каждые две кружки Джо прислушивается к своим ощущениям.
Если не менее двух кружек пива из последних трёх кружек разбавлены, то Джо разносит бар к чертям собачьим. 

\begin{enumerate}
    \item Сколько кружек пива в среднем успеет выпить Джо прежде чем разнесёт Огненную Зебру?
    \item Если все три последние кружки пива разбавлены, то Джо разносит не только Огненную Зебру, 
    но и всю прилежащую улицу. Какова вероятность данного сценария?
\end{enumerate}


\item Камала Харрис подбрасывает кубик до первого выпадения восьмёрки.
Все грани кубика выпадают равновероятно, однако на его шести гранях написаны числа 3, 4, 5, 6, 7, 8.
Дональд Трамп подбрасывает правильный октаэдр до выпадения восьмёрки. 
На гранях октаэдра написаны числа от 1, 2, 3, 4, 5, 6, 7, 8.

\begin{enumerate}
    \item Постройте гистограмму числа бросков кубика по $B = 10000$ экспериментов. 
    \item Оцените безусловное математическое ожидание числа бросков кубика.
    \item Оцените безусловную вероятность окончания игры быстрее, чем за 5 бросков кубика.
    \item Постройте условную гистограмму числа бросков октаэдра, если известно что грани 1 и 2 не выпадали. 
    Общее количество экспериментов здесь должно быть таким, чтобы число число экспериментов, где не выпадали грани 1 и 2 оказалось равным $B = 10000$.
    \item Оцените условное ожидание числа бросков октаэдра, если грани 1 и 2 не выпадали. 
    \item Оцените условную вероятность окончания игры быстрее, чем за 5 бросков октаэдра, если грани 1 и 2 не выпадали.
\end{enumerate}

\end{enumerate}


\end{document}

% здесь проектируемая часть

\end{document}

