
\documentclass[12pt]{article}

% \usepackage{physics}

\usepackage{hyperref}
\hypersetup{
    colorlinks=true,
    linkcolor=blue,
    filecolor=magenta,      
    urlcolor=cyan,
    pdftitle={Overleaf Example},
    pdfpagemode=FullScreen,
    }

\usepackage{tikzducks}

\usepackage{tikz} % картинки в tikz
\usepackage{microtype} % свешивание пунктуации

\usepackage{array} % для столбцов фиксированной ширины

\usepackage{indentfirst} % отступ в первом параграфе

\usepackage{sectsty} % для центрирования названий частей
\allsectionsfont{\centering}

\usepackage{amsmath, amsfonts, amssymb} % куча стандартных математических плюшек

\usepackage{comment}

\usepackage[top=2cm, left=1.2cm, right=1.2cm, bottom=2cm]{geometry} % размер текста на странице

\usepackage{lastpage} % чтобы узнать номер последней страницы

\usepackage{enumitem} % дополнительные плюшки для списков
%  например \begin{enumerate}[resume] позволяет продолжить нумерацию в новом списке
\usepackage{caption}

\usepackage{url} % to use \url{link to web}


\newcommand{\smallduck}{\begin{tikzpicture}[scale=0.3]
    \duck[
        cape=black,
        hat=black,
        mask=black
    ]
    \end{tikzpicture}}

\usepackage{fancyhdr} % весёлые колонтитулы
\pagestyle{fancy}
\lhead{}
\chead{}
\rhead{Домашние задания для самураев}
\lfoot{}
\cfoot{}
\rfoot{}

\renewcommand{\headrulewidth}{0.4pt}
\renewcommand{\footrulewidth}{0.4pt}

\usepackage{tcolorbox} % рамочки!

\usepackage{todonotes} % для вставки в документ заметок о том, что осталось сделать
% \todo{Здесь надо коэффициенты исправить}
% \missingfigure{Здесь будет Последний день Помпеи}
% \listoftodos - печатает все поставленные \todo'шки


% более красивые таблицы
\usepackage{booktabs}
% заповеди из докупентации:
% 1. Не используйте вертикальные линни
% 2. Не используйте двойные линии
% 3. Единицы измерения - в шапку таблицы
% 4. Не сокращайте .1 вместо 0.1
% 5. Повторяющееся значение повторяйте, а не говорите "то же"


\setcounter{MaxMatrixCols}{20}
% by crazy default pmatrix supports only 10 cols :)


\usepackage{fontspec}
\usepackage{libertine}
\usepackage{polyglossia}

\setmainlanguage{russian}
\setotherlanguages{english}

% download "Linux Libertine" fonts:
% http://www.linuxlibertine.org/index.php?id=91&L=1
% \setmainfont{Linux Libertine O} % or Helvetica, Arial, Cambria
% why do we need \newfontfamily:
% http://tex.stackexchange.com/questions/91507/
% \newfontfamily{\cyrillicfonttt}{Linux Libertine O}

\AddEnumerateCounter{\asbuk}{\russian@alph}{щ} % для списков с русскими буквами
\setlist[enumerate, 2]{label=\asbuk*),ref=\asbuk*}

%% эконометрические сокращения
\DeclareMathOperator{\pCorr}{\mathrm{pCorr}}
\DeclareMathOperator{\Cov}{\mathbb{C}ov}
\DeclareMathOperator{\Corr}{\mathbb{C}orr}
\DeclareMathOperator{\Var}{\mathbb{V}ar}
\DeclareMathOperator{\col}{col}
\DeclareMathOperator{\row}{row}

\let\P\relax
\DeclareMathOperator{\P}{\mathbb{P}}

\let\H\relax
\DeclareMathOperator{\H}{\mathbb{H}}

\DeclareMathOperator{\CE}{\mathrm{CE}}



\DeclareMathOperator{\E}{\mathbb{E}}
% \DeclareMathOperator{\tr}{trace}
\DeclareMathOperator{\card}{card}

\DeclareMathOperator{\Convex}{Convex}

\newcommand \cN{\mathcal{N}}
\newcommand \dN{\mathcal{N}}
\newcommand \dBin{\mathrm{Bin}}


\newcommand \RR{\mathbb{R}}
\newcommand \NN{\mathbb{N}}





\begin{document}

\section*{Домашнее задание 7}

% knad, fall 2024

Дедлайн: 2024-12-03, 23:59.

\begin{enumerate}

\item Пара величин $(X, Y)$ имеет функцию плотности $f(x, y) = 2x^3 + y$ на квадрате $[0;1] \times [0;1]$ и $0$ за его пределами. 
\begin{enumerate}
    \item Найдите условную функцию плотности $f(y \mid x)$.
    \item Найдите частные функции плотности $f_X(x)$ и $f_Y(y)$.
    \item Найдите функцию плотности $f_W(w)$ и функцию распределения $F_W(w)$ величины $W = X - Y$. 
    \item Найдите ожидание $\E(X + 5Y)$ и дисперсию $\Var(X + 5Y)$.
    \item Найдите совместную функцию плотности пары $(V = 2X + 3Y, W = X - Y)$. Аккуратно укажите область, где новая плотность положительна. 
    \item Найдите условное ожидание $\E(Y \mid X = x)$ и условную дисперсию $\Var(Y \mid X = x)$.
\end{enumerate}

\item Рассмотрим пуассоновский поток снежинок $(X_t)$ падающих на раскрытую ладошку с интенсивностью $\lambda = 0.5$ снежинок в секунду.
\begin{enumerate}
%    \item Я только что раскрыл ладошку. Какова вероятность того, что следующая снежинка упадёт раньше, чем через три секунды?
    \item Какова вероятность того, что за $5$ секунд на ладошку упадёт не менее двух снежинок?
    \item Я только что раскрыл ладошку. Какова вероятность того, что следующие две снежинки упадут раньше, чем через три секунды?
    \item Выпишите функцию плотности времени $T$ от раскрытия ладошки до выпадения третьей снежинки. 
    \item Найдите $\E(T)$ и $\Var(T)$.
    \item Выпишите функцию плотности отношения $R$ времени выпадения третьей снежинки к времени выпадения десятой снежинки. 
    \item Найдите $\E(R)$ и $\Var(R)$.
    \item Найдите вероятность $\P(X_{10} = 5 \mid X_{4} = 1)$.
    \item Найдите условные ожидание $\E(X_{10} \mid X_{4} = 1)$ и дисперсию $\Var(X_{10} \mid X_4 = 1)$.
\end{enumerate}

\item Страховые случаи наступают согласно пуассоновскому потоку с интенсивностью $100$ случаев в месяц. 
Выплаты по каждому страховому случаю распределены независимо от других случаев и времени наступления равномерно $0$ до $1$ ундециллиона рублей. 

Проведите $10^4$ симуляций этого процесса длиной в $1$ месяц. 

\begin{enumerate}
    \item Постройте гистограмму суммарных выплат за $10$ дней. 
    \item Оцените вероятность того, что за $10$ дней придётся выплатить более $12$ ундециллионов рублей. 
    \item Оцените размер резерва, необходимый страховой компании для того, чтобы за месяц вероятность исчерпания этого резерва была равна $0.05$.
    \item Как изменятся ответы на вопросы (б) и (в), если месяц начался с понедельника, а в субботу и воскресенье интенсивность страховых случаев падает до $10$ случаев в месяц?
\end{enumerate}


\end{enumerate}


\end{document}
